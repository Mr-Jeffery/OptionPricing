\documentclass{article}
\usepackage[utf8]{inputenc}

\title{FD method in Differential Equations}
\author{Mr-Jeffery}
\date{August 2022}

\begin{document}

\maketitle

\section{Finite Difference Method}
Finite difference method is an important computational method to mind the numerical solution of a differential equation. The main idea in to set up an equidistant grid of nodes on the domain of integration $[\alpha,\beta]$, such that 
$$x_i=\alpha+(i-1)*h\qquad(i=1,2,\cdot\cdot\cdot,n+1)$$ 
in which  
$h=\frac{(\beta-\alpha)}n$.

Consider a boundary value problem of second order:
$$u''+u' P(x)+u Q(x)=R(x)\\$$
$$u(\alpha)=a$$
$$u(\beta)=b$$
The basic idea is to approximate the first and derivative at each point by
$$u_i'=\frac{u_{u+1}-u_{i-1}}{2h}+O(h^2)$$
and
$$u_i''=\frac{u_{i+1}-2u_{i}+u_{i-1}}{h^2}+O(h^2)$$
then the differential equation can be written as

$$\frac{u_{i+1}-2u_{i}+u_{i-1}}{h^2}
    +\frac{u_{i+1}-u_{i-1}}{2h} P(x_i)
    +u_i Q(x_i)
    =R(x_i)\qquad(2\le i\le n)$$
$$u(x_1)=a$$
$$u(x_{n+1})=b$$
Then we can obtain the numerical solution by solving the linear equations above.

\section{American Put Option}

Shown by Black and Scholes (1973) and Merton (1973b), the partial differential equation governing the value of the option between dividend payment dates is
$$\frac12 \sigma^2 S^2 P_{SS} + r S P_{S} - r P - P_\tau=0 \quad S_f(\tau)<S,\quad0<tau<\tau_{max}$$
in which $\sigma^2$ is the instantaneous variance rate, P is the value of the American put option, $\tau$ is the remain time before expiration. According to Kim et al. (1990), the value of an American option at given time and price should be 
$$\qquad V(\tau,S(\tau))=v(\tau,S(\tau))+r\int_0^{\tau}K e^{-r(\tau-u)}\Phi(-d_{-}(\tau-u,S/S(u)))\mathrm{d}u$$
$$-q\int_0^{\tau}K e^{-q(\tau-u)}\Phi(-d_{+}(\tau-u,S/S(u)))\mathrm{d}u$$
where $v(\tau,S)$ is the European option price which can be calculated directly through formula, and the integrals are the early exercise premium terms.
The boundary condition of $S_f(\tau)$ is 
$$\qquad K-S_f(\tau)=V(\tau,S_f(\tau))$$
where the immediate payoff on the left hand side equals the value of the option.

Because of the $S_f(\tau)$ and $\tau$ inside the integral, this equation has no analytical solution. Thus the early exercise boundary can only be obtained through numerical solution.

In this research, we solve the equation by Richardson fixed point iteration by a universal grid $\{\tau_0,\tau_1,\cdot\cdot\cdot,\tau_{max}\}$, in which $\tau_0=0$ and $\tau_{max}$ is the longest maturity for which we shall need to recover the exercise boundary. With some mathematical technique , the equation can be converted into another form
$$S_f(\tau_i)=K e^{(q-r)\tau_{i}} \frac{N(\tau_i,S_f)}{D(\tau_i,S_f)}$$
where
$$N(\tau,S_f)=\Phi(-d_{-}(\tau,S_f(\tau)/K)+r\int_0^{\tau} e^{r u}\Phi(-d_{-}(\tau-u,S_f(\tau)/S(u)))\mathrm{d}u$$
$$D(\tau,S_f)=\Phi(-d_{+}(\tau,S_f(\tau)/K)+q\int_0^{\tau} e^{q u}\Phi(-d_{+}(\tau-u,S_f(\tau)/S(u)))\mathrm{d}u$$
The iteration formula at $j$th time is given by
$$S^{(j)}_f(\tau_i)=K e^{(q-r)\tau_{i}} \frac{N(\tau_i,S^{(j-1)}_f)}{D(\tau_i,S^{(j-1)}_f)}$$
Like mentioned above, the equation has no analytical solution, thus the integration inside the iteration scheme is conducted numerically by using trapezoid approximation.
$$\int_{\tau_{i-1}}^{\tau_{i}}f(u) \approx \frac{1}{2} (f(\tau_{i-1})+f(\tau_{i})) \Delta u$$
After a few iteration, $S_f$ converges and we should be able to get a stable result.
\section{Conclusion}
The core idea of finite difference method is to consider the solution as a set of points instead of a continuous curve, then the whole system is transformed into a linear system, which can be solved by linear algebra.
\end{document}
